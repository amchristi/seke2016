\documentclass{article}
\title{Proofs}
\begin{document}

Based on the assumptions mentioned in the paper and also from the fact that
delta debugging reduces failing test cases:
\begin{itemize}
\item totalfailed amd total passed are constants. Test suite remains unchanged 
and no changes are made to number of passing and failing test cases.
\item Delta debugging is set to reduce only failing test cases, thus passed(e) remains
unchanged.
\end{itemize}
\section{Tarantula}

\[
s(e) = \frac{\frac{f(e)}{TF}}{\frac{p(e)}{TP} + \frac{f(e)}{TF}}
\]
\[
TF = k_2 > 0 (mostly) k_1 = \frac{P(e)}{TP} > 0
\]
\[
y = \frac{\frac{x}{k_2}}{k_1 + \frac{x}{k_2}} = \frac{x}{k_2 k_1 + x}
\]

for some $x_2$ and $x_1$, $x_2 < x1$  but $y_2> y_1$
\[
\frac{x_2}{k_1 k_2+ x_2} > \frac{x_1}{k_2 k_1 + x_1}
\]
\[
x_2 x_1 + x_2 k_2 k_1 > x_1 k_2 k_1 + x_1 x_2
\]

$k_2 k_1 (x_2 - x_1) > 0$ but $x_2 < x_1$ $x_2 - x_1 < 0$ which is impossible.
Hence, s(e) is monotically increasing in f(e).

\section{Ochiai}
\[
s(e)  = \frac{f(e)}{\sqrt{TF(p(e)+f(e))}}
\]
$TF = k_1 > 0 $ $p(c) = k_2> 0$ $s(c) = y$ $f(e)=x$
$y=\frac{x}{\sqrt{k_1 (k_2 +x)}}$
for some $x_2 < x_1$,$ y_2 > y_1$

\[
\frac{x_2}{\sqrt{k_1 (k_2 + x_2)}}> \frac{x_1}{ \sqrt{k_1(k_2 + x_1)}}
\]


\[
\frac{x^{2}_2}{{k_1 (k_2 + x_2)}}> \frac{x^{2}_1}{k_1 (k_2 + x_1)}
\]

$x^{2}_2 k_2 + x^{2}_2 x_1 > x^{2}_1 k_2 + x^{2}_1 x_2$
$x_2 x_1  (x_2 - x_1) + k_2 (x^{2}_{2}-x^{2}_{1}) > 0$
which implies that $x_2 > x_1$.
Hence, $s(e)$ is monotically increasing in $f(e)$

\section{Jaccard}
$s(e) = \frac{f(e)}{f(e)+TF}$
$TF=k> 0$ $s(e)=y$ $f(e)=x$
$y=\frac{x}{x+k}$ $k > 0$

Proof by contradiction:
Assume that for some $x_2 < x_1$ and $y_2 > y_1$

\[
\frac{x_2}{x_2 + k} > \frac{x_1}{x_1 + k}
\]
\[
x_2 x_1 + x_2 k > x_1 x_2 + x_1 k
\]

$(x_2 - x_1) k > 0$ which implies that  $x_2 > x_1$
Hence, $s(e)$ is monotically increasing in f(e).

\end{document}