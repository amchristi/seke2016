Debugging is one of the most time-consuming and difficult aspects of
software development.  Recent years have seen a wide variety of
research efforts devoted to easing the burden of debugging by
automatically localizing faults.  The most popular approaches use
spectra of failing and successful executions to score program entities
according to how likely they are to be faulty.  Since the original
work on this topic, many formulas have been proposed to improve the
accuracy of scores.  Most of these improvements are either marginal or
context-dependent.  This paper proposes that independent of the
scoring method used, the effectiveness of spectra-based localization
can be dramatically improved by, when possible, delta-debugging
failing test cases and basing localization only on the reduced test
cases.  Unlike most formula changes, under reasonable assumptions
reduction can never reduce the effectiveness of localization.
Moreover, we show that for programs and faults taken from the
localization literature or from a large case study of Mozilla's
JavaScript engine, delta-debugging, localizing only after reduction
produces typically much better rankings for faults than localization
without reduction, independent of the localization formula used.
