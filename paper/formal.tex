\section{Assumptions and Guarantees}
\label{formal}

In addition to being orthogonal to the spectrum-based localization
formula used, test case reduction has a second major advantage
independent of empirical results.  Namely, under a set of assumptions
that hold in many cases, \emph{reduction can only improve, or leave
unchanged, the effectiveness of localization}.  All formulas for
localization have some instances in which they diminish the
effectiveness of localization compared to an alternative formula
\cite{yoo2014no}. For example, while Tarantula is not the most
effective formula in most studies, it is often best for an
individual fault and test suite, and may be more robust to
coincidentally correct tests than some other formulas \cite{CCT}.
Reducing failing tests before applying a formula, however, at worst
leaves the effectiveness of localization unchanged, for most of the
formulas in widespread use that we are aware of, under three
assumptions:

\begin{enumerate}
\item all failing test cases used in the localization involve the same
fault,
\item each failing test case reduces to a test case that fails due to the same fault as the original test case, and
\item reducing the input size (in components) also covers less code when the test executes.
\end{enumerate}

The first assumption is probably the least likely to hold in some
settings; however, it is also the assumption that is least relied
upon.  Reduction will only be harmful if some faulty code is removed
from some failing test cases that executed it but where it was
irrelevant to the fault.  While this is possible, it is only harmful
when the removal of non-faulty lines does not overcome this in
producing an overall ranking of statements.  That is, the localization
will suffer when, for some reason, faulty lines that are not required
for a given failure are removed at a \emph{higher rate} than non-faulty lines
that are not required for that failure.  The second assumption is a usual assumption of delta-debugging.  Most
delta-debugging setups are engineered with this goal in mind, often
using some aspect of failure output or test case structure to keep
``the same bug.''  Observed ``slippage'' rates for faults
seem to be fairly low \cite{PLDI13}, even with little mitigation.
Note that in the setting where a program has a single fault,
assumptions 1 and 2 always hold.  In a single fault setting, the
coverage vector for test cases for the faulty code (which must execute
for failure to take place) will be the same before and after reduction
of test cases.


Given these assumptions, we now show that reduction is, at worst,
harmless for most formulas.  Recall that spectrum-based localizations
rely on only a few values relevant to each entity $e$ to be ranked in
a localization: $\text{passed}(e)$, $\text{failed}(e)$,
$\text{totalpassed}$, and $\text{totalfailed}$.  Given assumptions 1-3
above, \emph{for faulty statements all of these formula elements will
be unchanged after delta-debugging}.  For non-faulty statements, the
only possible change is that $ \text{failed}(e)$ may be lower than
before failing test cases were reduced. Holding the other values
constant, it is trivial to show that most formulas under consideration
are (as we would expect), monotonically increasing in
$\text{failed}(e)$.  Therefore, after reduction, the suspiciousness
scores for faulty statements are unchanged and the suspiciousness
scores for non-faulty statements are either unchanged or lower.  The
rank of all faulty statements is therefore either unchanged or
improved.  There are many spectrum-based fault localization formulas
in use. In our evaluation, we have used 3 well known examples, in
addition to the basic Tarantula \cite{Jones2002} formula, as
representative:

{\bf Ochiai \cite{Ochai}:}
$$ \text{suspiciousness}(e) = \frac{\text{failed}(e)}{\sqrt{(\text{totalfailed}) (\text{failed}(e) + \text{passed}(e))}} $$

{\bf Jaccard \cite{Pinpoint}:}
$$ \text{suspiciousness}(e) = \frac{\text{failed}(e)}{\text{failed(e)} + \text{totalfailed}} $$

{\bf SBI: \cite{EmpirReduce,StatDebug}}
$$ \text{suspiciousness}(e) = \frac{\text{failed}(e)}{\text{failed(e)} + \text{passed}(e)} $$


All these formulas are monotonically increasing in
$\text{failed}(e)$.\footnote{Proofs available on request, and to be posted on our web page as an appendix.} Reduction can only theoritically improve fault localization or in worst case leave it unchanged if formula under consideration is monotonically increasing $\text{failed}(e)$. We chose these 4 formulas as they were used before to study effect of \emph{test suite reduction on fault localizaiton},to study \emph{effect of test case purificaiton on fault localization} \cite{EmpirReduce} \cite{PureTest}. One of the formula that we found not to be monotonically increasing in $\text{failed}(e)$ is AMPLE \cite{AMPLE}. The guarantee that reduction can only improve fault localization effectiveness does not hold for such a case. We note that according the the
theoretical analysis of Xie et al. \cite{Theoretical}, formulas making
number of failed test cases containing $e$ the primary
determinant of rank are best under certain assumptions, which fits
our expectations for the impact of reduction well.



