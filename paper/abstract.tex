Spectrum-based fault localization is one of the most popular and studied methods for automated debugging. Since the original work on this topic, many formulas have been proposed to improve the accuracy of fault localization scores. Many of these improvements are either marginal or context-dependent.  This paper proposes that, independent of the scoring method used, the effectiveness of spectrum-based localization can be dramatically improved by, when possible, delta-debugging failing test cases and basing localization only on the reduced test cases. We show that for programs and faults taken from the standard localization literature, a large case study of Mozilla’s JavaScript engine using 10 real faults, and mutants of various open-source projects, localizing only after reduction often produces much better rankings for faults than localization without reduction, independent of the localization formula used, and the improvement is often even greater than that provided by changing from the worst to the best localization formula for a subject.