\section{Experimental Results}
\label{sec:experiments}

\subsection{Evaluation Measure}

Because we aim to take into account the findings of Parnin and Orso
\cite{AutoHelp}, our evaluation of fault localizations is based on a
\emph{pessimistic absolute rank} of the highest ranked faulty
statement.  That is, for each set of suspiciousness metrics computed,
our measure of effectiveness is the \emph{worst possible position} at
which the first faulty statement can be reached, when examining the
code in suspiciousness-ranked order.
\footnote{We consider reaching \emph{any} faulty statement to be sufficient, as in \cite{NearNeighbor}.}
%\footnote{As usual since at least the work of Reneiris and Reiss\cite{NearNeighbor}, we consider reaching \emph{any} faulty statement to be sufficient.}  
For example,
if ten statements all receive a suspiciousness score of 1.0 (the
highest possible suspiciousness), and one of these is the fault, we
assign this localization a rank of 10; an unlucky programmer might
examine this statement last of the ten highest-ranked statements.
Pessimistic rank nicely distinguishes this result from another
localization that also places the bug at score 1.0, but gives twenty
statements a 1.0 score. In our view, following Parnin and Orso
\cite{AutoHelp}, the most important goal of a localization is to
direct the developer to a faulty statement as rapidly as possible,
ignoring the size of the entire program or even of the faulty
execution.  

\subsection{SIR Programs}

\begin{figure*}
  \centering
  \includegraphics[width=1.8\columnwidth]{siemens1}
  \caption{First Set of SIR Results (Log Scale)}
  \label{fig:allSIR1}
\end{figure*}

\begin{figure*}
  \centering
  \includegraphics[width=1.8\columnwidth]{siemens2}
  \caption{Second Set of SIR Results (Log Scale)}
  \label{fig:allSIR2}
\end{figure*}


\begin{table}
\begin{center}
\resizebox{\columnwidth}{!}{
\begin{tabular}{|c||c|c|c|c|c|}
\hline
Subject & Avg. & Avg. (DD) & \#Better & \#Same & \#Worse \\
\hline
\hline
{\tt print\_tokens} & 59.7 & 29.7 & 18 & 10 & 0  \\
\hline
{\tt print\_tokens2} & 27.0 & 5.8 & 19 & 17 & 4  \\
\hline
{\tt replace} & 24.5 & 21.8 & 37 & 76 & 11  \\
\hline
{\tt schedule} & 7.7 & 14.3 & 18 & 14 & 4  \\
\hline
{\tt schedule2} & 92.4 & 76.6 & 28 & 12 & 0 \\ 
\hline
{\tt tot\_info} & 29.7 & 17.7 & 75 & 16 & 1 \\
\hline
\hline
{\bf Total} & & & 195 & 145 & 20  \\
\hline
\end{tabular}%
}
\end{center}
%\vspace{-5mm}
\caption{SIR Fault Rank Change Result Frequencies}
\label{tab:avgimproved}
\vspace{-5mm}
\end{table}

Our initial experiments use the Siemens/SIR \cite{doESE05,Siemens}
suite programs studied in many previous papers on fault localization,
in particular the classic evaluation of the Tarantula technique
\cite{Tarantula}.  These subjects provide a large number of faults,
reasonable-sized test suites, and have historically been used to
evaluate localization methods.

Of the seven Siemens programs considered in the empirical evaluation
of Tarantula, only one was unsuitable for delta debugging: TCAS takes
as input a fixed-size vector of integers, and therefore its inputs
cannot be easily decomposed.  For the remainder of the programs, the
input is easily considered as either 1) a sequence of characters or 2)
a sequence of lines, when character-level delta debugging is not
efficient (and so unlikely to be chosen by users in practice), which
was required for the {\tt tot\_info} subject.  In all cases, reduction
took on average less than three seconds per failing test case, an essentially
negligible computational cost.



We evaluated our proposal by 1) first computing the fault ranking for
each version of each subject by the five formulas then 2) performing
the same computation, but using only reduced (by delta-debugging)
versions of the failing tests.  Reduction was performed using Zeller's
delta-debugging scripts, available on the web, and comparing the output
of the original (correct) version of the program and the faulty
version as a pass/fail oracle.

Figures \ref{fig:allSIR1} and \ref{fig:allSIR2} show the
results
%\footnote{Version 32 of {\tt replace} is missing because the fault relies on the library definition of {\tt isalnum}, and is no
%longer present on modern Linux systems, as we determined after
%communicating with Gregg Rothermel and the SIR maintainers.}.  
The
lighter shaded bars show the ranking of the fault, without any
reduction.  The darker bars show the ranking after delta-debugging all
failures.  The graphs are shown in log-scale due to the range of
rankings involved.  In many cases, reducing test cases before
localizing improved the ranking of the fault by a factor of two or
more.  Results for individual subject vary: for {\tt print\_tokens},
the average ranking for faults, over all bugs and all formulas, is
59.5 without reduction, and 36.4 with reduction.  The result is
improved by reduction in 19 cases, remains the same in 15 cases, and
is worse in 1 case.  These numbers improve to 29.8 average ranking, 18
improvements, and 10 unchanged results.
Table \ref{tab:avgimproved} shows similar data for all the SIR
subjects.  

Our results exhibits improvement in fault rank was 1.3 times as common as no change
in rank, and \emph{nearly 10 times as common as worse rank for the
fault.}  In addition to the frequency of improvement of fault rank, it
is also important to examine the degree of improvement (or the
opposite) provided by reduction.  Table \ref{tab:rankchange} shows,
the min, max, and average for changes in rank.  the effect
size when reduction improved rank was usually much larger than the
effect size when it gave worse results.  For {\tt replace}, the
subject with the most instances where reduction made fault ranking
worse, we see that the effect size when reduction was harmful was much
smaller than when reduction was helpful.  Furthermore, when reduction
helped, it often improved the ranking of the fault by more than
\emph{optimally} switching formula  That is, we can ask: if we compare
taking the \emph{worst} formula and applying reduction to improve the
localization, how often is this better than switching to the
\emph{best} localization formula for that subject and fault?
Obviously applying reduction is more practical, since we don't know in
advance which formula will perform best, until we know the correct
result.  By this comparison, it was better to
apply reduction than switch from \emph{worst} to \emph{best} formula
in 36 cases over all SIR subjects;  it was better to switch formula in
only 22 cases.



\begin{table}
\begin{center}
\resizebox{\columnwidth}{!}{%
\begin{tabular}{|c||c|c|c||c|c|c|}
%\hline
\hline
& \multicolumn{3}{|c|}{Better} & \multicolumn{3}{|c|}{Worse} \\
Subject & Min & Max & Avg. & Min & Max & Avg \\
\hline
\hline
{\tt print\_tokens} & 6 & 85 & 44.6 & N/A & N/A & N/A \\
\hline
{\tt print\_tokens2} & 3 & 67 & 45.8 & 6 & 6 & 6.0 \\
\hline
{\tt replace} & 1 & 30 & 9.6 & 1 & 3 & 1.4 \\
\hline
{\tt schedule} & 1 & 15 & 4.8 & 85 & 75 & 81.3 \\
\hline
{\tt schedule2} & 4 & 35 & 22.6 & N/A & N/A & N/A \\
\hline
{\tt tot\_info} & 1 & 81 & 14.7 & 3 & 3 & 3.0 \\
\hline
%\hline
\end{tabular}%
}
\end{center}
\caption{SIR Fault Rank Change Effect Sizes}
\label{tab:rankchange}
\end{table}


\subsection{SpiderMonkey JavaScript Engine}
SpiderMonkey is the JavaScript Engine for Mozilla, an extremely widely
used, security-critical interpreter/JIT compiler.  SpiderMonkey has
been the target of aggressive random testing for many years now.  A
single fuzzing tool, \texttt{jsfunfuzz} \cite{jsfunfuzz}, is
responsible for identifying more than 1,700 previously unknown bugs in
SpiderMonkey \cite{jsfunfuzzbugs}.  SpiderMonkey is (and was) very
actively developed, with over 6,000 code commits in the period from
1/06 to 9/11 (nearly 4 commits/day).  SpiderMonkey is thus ideal for
evaluating how reduction aids localization when using a sophisticated
random testing system, using the last public release of the
\texttt{jsfunfuzz} tool \cite{jsfunfuzz}, modified for swarm testing \cite{ISSTA12}.
Using a set of faults in SpiderMonkey version 1.6 found with random
testing in previous research \cite{PLDI13}, we show that reduction is
essential for localization of bugs found using random testing, and
that the use of reduction is even somewhat more important than the
choice of localization formula.  Note that this is an extremely
challenging setting for spectrum-based localization: randomly
generated successful tests tend to cover a wide variety of behavior,
some of it in a very shallow fashion, which makes distinguishing the
signal of lines covered more frequently in failing tests from general
noise (and particularly the noise of lines that are simply hard to
cover and happen to appear in the failures) very difficult.

\begin{table}
\begin{center}
\begin{tabular}{|c||c|c|c|c|c||c|c|c|c|c|}
\hline

\hline
Bug\# & Revision Fixed &  \#Failures & {\tt diff} size \\
\hline
\hline
{\tt R60} & 1.16.2.1 & 1 & 115  \\
\hline
{\tt R95} & 1.3.2.3.8 & 7 & 111   \\
\hline
{\tt R115} & 1.4.8.1 & 4 & 592   \\
\hline
{\tt R360} & 3.117.2.6 & 3 & 223  \\
\hline
{\tt R880} & 3.17.2.14 & 28 & 272   \\ 
\hline
{\tt R1172} & 3.208.2.63 & 150 & 214  \\
\hline
{\tt R1294} & 3.241.2.1 & 405 & 80  \\
\hline
{\tt R1543} & 3.36.16.1 & 146 & 169  \\
\hline
{\tt R1561} & 3.37.2.1.4.1.2.2 & 2 & 31  \\
\hline
{\tt R1873} & 3.50.2.29 & 1,041 & 56  \\
\hline
\hline
\end{tabular}
\end{center}
\caption{Spidermonkey Bugs}
\label{tab:spiderbugs}
\end{table}


\begin{figure*}[t]
  \centering
  \includegraphics[width=2\columnwidth]{naspidermonkey}
 \vspace{-2.2in}
  \caption{SpiderMonkey Results}
  \label{fig:spidermonkey}

\end{figure*}

Figure \ref{fig:spidermonkey} shows the change in rankings of the
faulty code for 10 SpiderMonkey bugs (Table \ref{tab:spiderbugs}).
These bugs were taken from our PLDI 2013 data set \cite{PLDI13}.  Out
of the 28 bugs studied in that paper we chose 10 random bugs for which, by
hand, we could confirm the true set of faulty lines in the code
commit.  Each bug is identified by the revision number of the commit
in which it was fixed: e.g., R0 maps to revision 1.10.4.1, the first
commit of Spidermonkey changes under consideration. Table
\ref{tab:spiderbugs} shows all bugs studied, the commit version fixing
the bug, the number of failing test cases for that bug (\# Failures),
and the size (in lines) of the fixing commit's {\tt diff}.  The faults
under consideration here are clearly non-trivial (in fact, most fixes
involved changes to multiple source files).  For our localization we
used the original and reduced test cases from PLDI 13 \cite{PLDI13}
plus a set of 720 randomly generated passing tests generated using the same method as in
the original data set.

Across these 10 bugs, the average ranking for the first faulty line
encountered was 1,550.7 without reduction, improving to 994.5 with
reduction.  Reduction improved the localization in 33 cases, with a
minimum improvement of 1 ranking and a maximum improvement of 2,137
positions.  The average improvement was 674 positions.  The results
were unchanged in 7 cases. It is important to note that even with such a challenging setting and real bugs, fault localization with reduction performs better then fault localization without reduction irrespective of formula being used, in worst case it performed as good. It was better to use reduction than to optimally (from worst to
best) switch formula for 5 of the 10 bugs; it was better to switch
formula in 4 cases, and in one case both methods gave the same result.
While the results show that reduction was extremely effective in
improving localization, it is also true that the localization was
\emph{still not very helpful} in many of these cases, considering absolute pessimistic rank as success criteria.  Of course,
SpiderMonkey 1.6 has over 80KLOC, and even reduced failing tests
typically executed over 8,000 lines of code, so a ``poor''
localization may be useful in such a large fault search space.  For 6
of the 10 bugs, all scores after reduction gave the fault a ranking of
128 or better.  We also suspect that experienced developers could dismiss
at least some of the lines in the rankings immediately, as we discuss
next.

In order to expand on how reduction improves localization, we examine in more detail the bug
we call ``R1543,'' fixed by commit 3.36.16.1, and its Tarantula
localization.  In our experiments, there were 146 test cases that
failed due to bug R1543.  The relevant commit contains three kind of
changes: (1) a rename refactoring, (2) a refactoring for readability,
and (3) the actual fault fix.  The fault is in {\tt jsexn.c}, in the
function {\tt Exception}. Comparing diffs of R1542 and R1543 we were
able to identify lines 583, 587, 589, 590 and 594 as the actual faulty
code. Before reduction, these lines (all with suspiciousness 0.716605)
had rankings from 191 on (ranking is arbitrary among buggy lines of
same suspiciousness).  After reduction, the rankings improved to start
at 98.  The suspiciousness values of faulty lines, as discussed in Section
\ref{formal}, remained unchanged.  However, many higher ranked non-faulty lines
became less suspicious after being removed from failing tests.  The
average coverage for an unreduced failing test was 11,461 lines, which
decreased to 8,349 (an average decrease of 3,112 lines) after reduction.

When we looked at the lines ranked before the faulty lines, we found
that 70 lines in unreduced ranking and 56 lines in post-reduction
rankings were assigned suspiciousness values of 1.0. These lines
execute only in failing test cases. Our further investigation suggests
that many of these lines are actually failure handling and reporting
code.  For example, 23 of the highly ranked lines for R1543 are part
of the functions {\tt my\_ErrorReporter} and {\tt Quit} in {\tt js.c},
clearly failure handling and reporting code that would not slow an
experienced SpiderMonkey developer in reaching the actual fault.  We
found such failure handling and reporting code in highly ranked lines
for other bugs as well (finding them is fairly easy, as they have
suspiciousness 1.0 by Tarantula).  While many lines that are not such
obviously non-faulty code remain for many SpiderMonkey bugs, this
suggests a follow-on to Parnin and Orso's experiments \cite{AutoHelp}:
while they included some more experienced developers in their study,
they did not investigate how developers highly skilled with \emph{a
particular code base} use fault localization.  We speculate that while
the raw rankings for many SpiderMonkey bugs are not very high, some of
these results could be more useful than is apparent, in the hands of
expert developers.  Such users are the most likely eventual industrial
adopters of localization, we believe, for complex projects where
debugging can be extremely difficult and time consuming and aggressive
automated testing is used.\footnote{We the authors have submitted
Chrome JavaScript engine bugs found during testing research and seen
that it may take months to understand and fix some complex failures.}


\subsection{Open Source Projects}
\label{sec:opensource}



\begin{table}
{\scriptsize
\begin{center}
\begin{tabular}{|c||c|c|c|c|c||c|c|c|c|c|}
\hline
\hline
& \multicolumn{3}{|c|}{Program Source} & \multicolumn{1}{|c|}{Test Suite} \\
\hline
Subject & \#Classes. & \#Methods & SLOC & \#Test cases \\
\hline
\hline
{\tt Apache Commons} & & & & \\
{\tt Validator} & 64 & 578 & 6,033 & 434 \\
\hline
{\tt JExel 1.0.0} & & & & \\
{\tt beta 13} & 43 & 133 & 1,522	 & 344  \\
\hline
{\tt JAxen} & 167 & 1,078 & 12,462 & 2,138\\
\hline
{\tt JParser} & 115 & 178 & 3,046 & 647 \\
\hline
{\tt Apache Commons} & & & & \\
{\tt CLI} & 23 & 208 & 2,667 & 364 \\ 
\hline
\hline
\end{tabular}
\end{center}
}
\caption{Open Source Subject Programs}
\label{tab:opensourcesubs}
\end{table}



The SIR results show that reduction is useful for localization in an
idealized setting where suites are relatively complete and faults are
chosen for certain properties, and the Spidermonkey results show its
value in a particularly challenging setting (where passing tests have
very wide-ranging coverage).  How does reduction affect localization
in a typical open source project?

We attempted to answer this question by picking five open source Java
programs (shown in Table \ref{tab:opensourcesubs}) and generating mutants for each of the project to simulate bugs as it is commonly done in recent fault localization literature.\cite{mutant} \cite{PureTest}



\begin{figure}[t]
  \centering
  \includegraphics[width=\columnwidth]{opensource1}
  \caption{First Set of Open Source Results}
  \label{fig:opensource1}
\end{figure}

\begin{figure}[t]
  \centering
  \includegraphics[width=\columnwidth]{opensource2}
  \caption{Second Set of Open Source Results}
  \label{fig:opensource2}
\end{figure}

 
Our strategy was to create mutants in the same way as Xuan and
Monperrus \cite{PureTest} in their own localization work, using 6
mutant operators.  From each set of mutants generated, we selected 5
or 6 mutants at random that met the following criteria: (1) the mutant
was killed by at least one test case and (2) the mutant generated no
errors in JUnit test cases.  A JUnit failure is caused by an
unsatisfied assertion, but an error is caused by another kind of test
failure, which may include some test setup or oracle problems.  Using
assertion failures only assures that we were remaining within the
intent of the original tests.

Figures \ref{fig:opensource1} and \ref{fig:opensource2} show the
results of applying reduction to these simulated faults.  Taking all
the open source projects and mutants together, we note that reduction
improved fault ranking in 51 cases, left it unchanged in 55 cases, and
made it worse in 2 cases.  The average improvement was 17.62 ranking
positions; the average negative effect size was 2 ranking positions.
The best improvement was 100
rank positions.  The average fault ranking without
reduction was 37.64 and with reduction it improved to 29.36.  


\subsection{Threats to Validity}

The primary threats to validity here are to external validity
\cite{Threats}, despite our use of a reasonable number of faults and subjects.  Our subjects are all C or
Java programs, for example, and for the open source projects we used
seeded rather than real faults.  To avoid
construct threats, we developed independent experimental code-bases
for some of the subjects, executed both, and compared results to
cross-check the shared code base used for all subjects.  Fortunately,
most tasks here are straightforward (test execution, coverage
collection, delta-debugging, and calculation of scores).

A second point (not strictly a threat) is that this paper focuses on
single-fault localization.    Even when
there are multiple faults, it may be better to use techniques for
clustering test cases by likely fault 
\cite{Jones07,PLDI13,Podgurski03,Podgurski04} and then perform
single-fault localization than to try to localize multiple faults at
once.
