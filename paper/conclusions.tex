\section{Conclusions}

Our primary conclusion is that, when possible, anyone attempting to
use spectrum-based fault localization should use delta-debugging to
\emph{reduce before localizing}.  Across Siemens subjects, real
Mozilla SpiderMonkey bugs, and mutants of a set of open source
projects, reducing test cases before localizing was seldom harmful and
in the cases where it caused harm the effect size was much smaller
than in the cases where reduction was helpful.  In most cases,
reduction was helpful, and it was sometimes extremely effective,
improving fault ranking by a factor of 2 (or more) and a very large
absolute rank, sometimes hundreds of lines. This makes sense: if
failing test cases only contained faulty code, fault localization
would be trivial.  Delta-debugging, by (usually) reducing the coverage
of non-faulty code, approaches this ideal situation as best we know
how at present.  While delta-debugging is not a panacea for
localization, in that it does not apply to some kinds of inputs and is
sometimes not helpful, it often produces a very large improvement in
localization effectiveness, quite often \emph{more so than can be
gained by switching from worst to best formula}.  We speculate that
reduction should also assist mutation-based fault localization methods
\cite{MUSE,multilingual,compilemutant}, since the mutants
that drive localization will be those that cause failing tests to
succeed, and reduction should limit these as well.

Our larger take-away message is that the lessons of Parnin and Orso
\cite{AutoHelp} should be taken to heart: rather than seek incremental
improvements in localization effectiveness, we need large improvements
in fault rank, and need to exploit all sources of information, not
just coverage vectors.  Even when reduction does not assist
localization, we believe that the reduced test cases are highly
valuable debugging aids.  Furthermore, because no single formula is
``best'' for all faults \cite{yoo2014no}, there is much to be gained
by devising aids to fault localization that apply to any formula and
any type of spectrum.  If automated fault localization is to be
adopted in real-world settings, we need more than a competing set of
ranking algorithms: we need a complete ecosystem for localization and debugging.
